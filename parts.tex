\section{Parts of Manuscript} 

This section describes the normal structure of a manuscript and how each part should be handled. The appropriate vertical spacing between various parts of this document is achieved in LaTeX through the proper use of defined constructs, such as \verb|\section{}|. 

%%-----------------------------------------------------------
\subsection{Title and Author Information} 
\label{sect:title}
The article title appears left justified at the top of the first page. The title font is 16 pt., bold. The rules for capitalizing the title are the same as for sentences; only the first word, proper nouns, and acronyms should be capitalized. Do not begin titles with articles (for example, a, an, the) or prepositions (for example, on, by, etc.). The word “novel” should not appear in the title, as publication will imply novelty. Avoid the use of acronyms in the title, unless they are widely understood. Appendix A contains more about acronyms.

The list of authors immediately follows the title, 18 points below. The font is 12 pt., bold and the author names are left justified. The author affiliations and addresses follow the names, in 10-pt., normal font and left justified. For multiple affiliations, each affiliation should appear on a separate line. Superscript letters (a, b, c, etc.) should be used to associate multiple authors with their respective affiliations. Corresponding author should be listed below the Keywords, following the line ``Address all correspondence to:"; include mailing address, telephone, fax, and E-mail address.

%%-----------------------------------------------------------
\subsection{Abstract} 
The abstract should be a concise summary of the paper. Because the abstract may be used in abstracting journals, it should be self-contained (that is, no numerical references) and substantive in nature, presenting concisely the objectives, methodology used, results obtained, and their significance. It should not exceed 200 words. For further guidelines, please read the brief article titled ``How to Write an Abstract (PDF)," \\(\linkable{http://spie.org/Documents/Publications/How to Write an Abstract.pdf}) by Philip Koopman. (Courtesy of Philip Koopman, Carnegie Mellon University.) Note the underlined link must be written on a single line in order for the link to function. Thus, a break in the line, created with {\verb+\\+} is required before the link or the preceding text should be editted to gracefully place the link on a single line. This type of adjustment is best done at end of the manuscript-preparation process.

%%-----------------------------------------------------------
\subsection{Keywords} 
Up to eight keywords should be specified. 

%%-----------------------------------------------------------
\subsection{Body of Paper} 
The body of the paper consists of numbered sections that present the main findings. These sections should be organized to best present the material.

To provide transition elements in your paper, it is important to refer back (or forward) to specific sections. Such references are made by indicating the section number, for example, ``In Sec.\ 2 we showed..." or ``Section 2.1 contained a description..." If the word Section, Reference, Equation, or Figure starts a sentence, it is spelled out. When occurring in the middle of a sentence, these words are abbreviated Sect., Ref., Eq., and Fig. 

At the first occurrence of an acronym, spell it out followed by the acronym in parentheses, for example, charge-coupled diode (CCD).

%%-----------------------------------------------------------
\subsection{Footnotes} 
Use textual footnotes only when necessary to present important documentary or explanatory material whose inclusion in the text would be distracting.\footnote{Example of a footnote.} Due to problems with HTML display, use of footnotes should be avoided. If absolutely necessary, the footnote mark must come at the end of a sentence. To insert a footnote, use the {\verb|\footnote{}|} command.

%%-----------------------------------------------------------
\subsection{Appendices} 
SPIE journals do not accept supplementary materials. However, it is acceptable to include an Appendix when necessary,  details such as derivations of equations, proofs of theorems, and details of algorithms. Equations and figures appearing in appendices should continue sequential numbering from earlier in the paper.

%%-----------------------------------------------------------
\subsection{Acknowledgments} 
In the acknowledgments section, which appears just before the references, the authors may credit others for their guidance or help. Also, funding sources or sponsorship information may be stated. The acknowledgments section does not have a section number.

%%-----------------------------------------------------------
\subsection{References} 
The References section lists books, articles, and reports that are cited in the paper. This section does not have a section number. The references are numbered in the order in which they are cited. Examples of the format to be followed are given at the end of this document.

The reference list at the end of this document is created using BibTeX, which looks through the file {\ttfamily report.bib} for the entries cited in the LaTeX source file.  The format of the reference list is determined by the bibliography style file {\ttfamily spiejour.bst}, as specified in the \\ \verb|\bibliographystyle{spiejour}| command.  Alternatively, the references may be directly formatted in the LaTeX source file.

For books \cite{Lamport94,Alred03,Goossens97} the listing includes the list of authors (initials plus last name), book title (in italics), page or chapter numbers, publisher, city, and year of publication.  Journal-article references \cite{Metropolis53,Harris06} include the author list, title of the article (in quotes), journal name (in italics, properly abbreviated), volume number (in bold), inclusive page numbers or citation identifier, and year.  A reference to a proceedings paper or a chapter in an edited book \cite{Gull89a} includes the author list, title of the article (in quotes), conference name (in italics), editors (if appropriate), volume title (in italics), volume number if applicable (in bold), inclusive page numbers, publisher, city, and year.  References to an article in the SPIE Proceedings may include the conference name, as shown in Ref.~\citenum{Hanson93c}.

The references are numbered in the order of their first citation. Citations to the references are made using superscripts, as demonstrated in the preceding paragraph. One may also directly refer to a reference within the text, for example, ``as shown in Ref.~\citenum{Metropolis53} ..."  Two or more references should be separated by a comma with no space between them. Multiple sequential references should be displayed with a dash between the first and last numbers \cite{Alred03,Perelman97,Lamport94,Goossens97,Metropolis53}. 

%%---------------------------
\subsubsection{Reference linking and DOIs} 
A Digital Object Identifier (DOI) is a unique alphanumeric string assigned to a digital object, such as a journal article or a book chapter, that provides a persistent link to its location on the internet. The use of DOIs allows readers to easily access cited articles. Authors should include the DOI at the end of each reference in brackets, if a DOI is available. See examples at the end of this manuscript. A free DOI lookup service is available from CrossRef at \\\linkable{http://www.crossref.org/freeTextQuery/}. The inclusion of DOIs will facilitate reference linking and is highly recommended. 

In the present LaTeX template, the author needs to add the DOI reference by including it in a ``note" in the bibliography file, as shown in the file {\verb+report.bib+}, for example, \\ {\verb+note = "[doi:10.1117/12.154577]"+}. The DOI may be used by the reader to locate that document with the link: {\verb+http://dx.doi.org10.1117/12.154577+}. 

%%-----------------------------------------------------------
\subsection{Biographies} 
A brief professional biography not to exceed 75 words may be provided for each author, if available. Biographies should be placed at the end of the paper, after the references. Personal information such as hobbies or birthplace/birthdate should not be included. Author photographs are not published.
