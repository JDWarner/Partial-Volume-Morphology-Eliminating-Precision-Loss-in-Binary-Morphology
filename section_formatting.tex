\section{Section Formatting} \label{sect:sections}

In LaTeX, a new section is created with the \verb|\section{}| command, which automatically numbers the sections. Sections will be numbered sequentially, starting with the first section after the abstract, except for the acknowledgments and references. (Note that numbering of section headings is not required, but the numbering must be consistent if used.) All section headings should be left justified.

Main section headings are in 12-pt., bold font, left-justified and in title case, where important words are capitalized.

Paragraphs that immediately follow a section heading are leading paragraphs and should not be indented, according to standard publishing style. The same goes for leading paragraphs of subsections and sub-subsections. Subsequent paragraphs are standard paragraphs, with 0.2-in (5 mm) indentation. There is no additional space between paragraphs. In LaTeX, paragraphs are separated by blank lines in the source file. Indentation of the first line of a paragraph may be avoided by starting it with \verb|\noindent|.

%%-----------------------------------------------------------
\subsection{Subsection Headings} 
All important words in a subsection (level 1) header are capitalized. Subsection numbers consist of the section number, followed by a period, and the subsection number within that section, without a period at the end. The heading is left justified and its font is 12 pt., italic.

%%-----------
\subsubsection{Sub-subsection headings} 
The first word of a sub-subsection is capitalized. The rest of the text is not capitalized, except for proper names and acronyms (the latter should only be used if well known). The heading is left justified and its font is 11 pt., italic. 
