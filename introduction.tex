\section{Introduction}
\label{sect:intro}  % \label{} allows reference to this section
This document shows the required format and appearance of a manuscript prepared for SPIE journals. Formatting guidelines must be carefully followed. Authors are advised to print this sample manuscript and use it as a reference while preparing their own paper, to ensure all guidelines are met.

%%-----------------------------------------------------------
\subsection{Use of This Document}

This document is prepared using LaTeX2e \cite{Lamport94,Goossens97} with the class file {\ttfamily spieman.cls}.  The LaTeX source file used to create this document is {\ttfamily article.tex}, which contains important formatting information embedded in it. Authors may use it as a template to create their own manuscript. While LaTeX properly handles most formatting issues, the author may occasionally need to intervene to obtain a satisfactorily  formatted manuscript.

%%-----------------------------------------------------------
\subsection{English}

Authors are strongly encouraged to follow the principles of sound technical writing, as found in Refs.~\citenum{Alred03} and \citenum{Perelman97}, for example. In addition, good English usage is essential. Authors whose native language is not English may wish to collaborate with a colleague whose English skills are more advanced. Alternatively, you may wish to have your manuscript professionally edited prior to submission by Editage, our recommended independent editorial service: \linkable{http://spie.org/EnglishEditing}. SPIE authors will receive a 10\% discount off their services. A spell checker can be helpful to discover misspelled words, but authors should also proofread their papers carefully prior to submission. Manuscripts that do not meet acceptable English standards or lack clarity may be rejected.

%%-----------------------------------------------------------
\subsection{Page Setup and Fonts}

All text and figures, including footnotes, must fit inside a text area 6.5 in.\ wide by 9 in.\ high (16.51 by 22.86 cm). Manuscripts must be formatted for US letter paper, on which the margins should be 1.0 in.\ (2.54 cm) on the top, 1 in.\ on the bottom, and 1 in\ on the left and right. 

The Times New Roman font is used throughout the manuscript, in the sizes and styles shown in Table~\ref{tab:fonts}. If this font is not available, use a similar serif font. The manuscript should not contain headers or footers. Pages should be numbered.

%% Use of [h] in following command forces table to appear "here", instead of a top or bottom of page, which is generally preferred.
\begin{table}
\caption{Fonts sizes and styles.} 
\label{tab:fonts}
\begin{center}       
\begin{tabular}{ll} %% this creates two columns
%% |l|l| to left justify each column entry
%% |c|c| to center each column entry
%% use of \rule[]{}{} below opens up each row
\hline 
Document entity & Brief description  \\
Article title & 16 pt., bold, left justified  \\
Author names & 12 pt., bold, left justified   \\
Author affiliations & 10 pt., left justified   \\
Abstract & 10 pt.  \\
Keywords & 10 pt.  \\
Section heading & 12 pt., bold, left justified  \\
Subsection heading & 12 pt., italic, left justified  \\
Sub-subsection heading & 11 pt., italic, left justified  \\
Normal text & 12 pt. \\
Figure and table captions &  10 pt. \\
\hline 
\end{tabular}
\end{center}
\end{table} 
